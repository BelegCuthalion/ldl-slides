\documentclass[handout]{beamer}

\usetheme[numbering=fullbar]{focus}
% Add option [numbering=none] to disable the footer progress bar
% Add option [numbering=fullbar] to show the footer progress bar as always full with a slide countx


\definecolor{main}{RGB}{55, 135, 177}
\definecolor{text}{RGB}{15, 40, 60}
\definecolor{background}{RGB}{255, 255, 255}

\setbeamertemplate{block begin}
{
  \par\vskip\medskipamount%
  \IfStrEq{\insertblocktitle}{}{}{
      \begin{beamercolorbox}[colsep*=.75ex]{block title}
        \usebeamerfont*{block title}\insertblocktitle%
      \end{beamercolorbox}%
  }
  {\parskip0pt\par}%
  \ifbeamercolorempty[bg]{block title}
  {}
  {\ifbeamercolorempty[bg]{block body}{}{\nointerlineskip\vskip-0.5pt}}%
  \usebeamerfont{block body}%
  \begin{beamercolorbox}[colsep*=.75ex,vmode]{block body}%
    \ifbeamercolorempty[bg]{block body}{\vskip-.25ex}{\vskip-.75ex}\vbox{}%
}


\setbeamercovered{transparent}

%------------------------------------------------

\usepackage{booktabs} % Required for better table rules
\usepackage{xstring}
\usepackage{tikz}
\renewcommand{\baselinestretch}{1.3}
\usepackage[T1]{fontenc}
\usepackage[sfdefault]{AlegreyaSans}
\usepackage{bussproofs}
\EnableBpAbbreviations
\usepackage{multicol}

\title{Logic of Dynamic Locales}

\author{Alireza Mahmoudian\\ \small (Joint work with A. A. Tabatabai M. Alizadeh)\\ University of Tehran\\ School of Mathematics, Statistics and Computer Science\\ amahmoudian@ut.ac.ir}

% \titlegraphic{\includegraphics[scale=.2]{res/ut-logo.png}}

\institute{}

\date{The Institute for Research in Fundamental Sciences (IPM), Tehran\\ May 2024}


\begin{document}


\begin{frame}
	\maketitle
\end{frame}

\def\secSTL{System $\mathbf{LDL}$, motivation and design}
\def\subImprDef{What is \emph{impredicativity}?}
\def\subBHK{Impredicativity in the BHK Interpretation}
\def\subIdea{Basic idea}
\def\subSTL{A logical system}
\def\secSemantics{Semantics}
\def\subAlgSemantics{Algerbraic Semantics}
\def\subExtensions{Extensions}
\def\subKripSemantics{Kripke Semantics}
\def\secProposal{Our contribution}
\def\subAnalytic{Cut-elimination}
\def\subGSTL{Equivalent analytic logical systems}
\def\subResults{Results}
\def\subFurther{Further works}

\begin{frame}{Agenda}
	\footnotesize
	\begin{enumerate}
		\begin{columns}
			\column{0.5\textwidth}
		\item \secSTL
		\begin{enumerate}
			\item \subImprDef
			\item \subBHK
			\item \subIdea
			\item \subSTL
		\end{enumerate}
		\item \secSemantics
		\begin{enumerate}
			\item \subAlgSemantics
			\item \subExtensions
			\item \subKripSemantics
		\end{enumerate}
			\column{0.5\textwidth}
		\item \secProposal
		\begin{enumerate}
			\item \subAnalytic
			\item \subGSTL
			\item \subResults
			\item \subFurther
		\end{enumerate}
	\end{columns}
	\end{enumerate}
\end{frame}

\section{\secSTL}

\begin{frame}{\subImprDef}
	\begin{block}{}
		``A definition is impredicative if it defines a member of a totality in terms of that totality itself.'' \cite{van2017predicativity}
		\vspace{1ex}
	\end{block}
	BHK Interpretation defines a proof for an intuitionistic proposition inductively. So a proof for
	\begin{itemize}
		\small
		\item an atomic proposition is supposed to be known from context,
		\item $\bot$ does not exist,
		\item $P \vee Q$ is either $\langle 0, a \rangle$, where $a$ is a proof of $P$, or $\langle 1, a \rangle$, where $a$ is a proof of $Q$,
		\item $P \wedge Q$ is $\langle a, b \rangle$, where $a$ is a proof of $P$ and $b$ is a proof of $Q$,
		\large
		\item $P \rightarrow Q$ is a function $f : p(P) \rightarrow p(Q)$, where $p(X)$ is the set of all proofs of the proposition $X$.
	\end{itemize}
\end{frame}

\begin{frame}{\subBHK}
	\begin{itemize}
		\small
		\item an atomic proposition is supposed to be known from context,
		\item $\bot$ does not exist,
		\item $P \vee Q$ is either $\langle 0, a \rangle$, where $a$ is a proof of $P$, or $\langle 1, a \rangle$, where $a$ is a proof of $Q$,
		\item $P \wedge Q$ is $\langle a, b \rangle$, where $a$ is a proof of $P$ and $b$ is a proof of $Q$,
		\large
		\item $P \rightarrow Q$ is a function $f : p(P) \rightarrow p(Q)$, where $p(X)$ is the set of all proofs of the proposition $X$.
	\end{itemize}
	In the clause for $P \rightarrow Q$, we are quantifying over \emph{all} proofs, but the set of \emph{all} proofs is just being defined.
\end{frame}

\begin{frame}{\subBHK}
	Consider the \emph{modus ponens} rule:
	\begin{prooftree}
		\AXC{$p : P$}
		\AXC{$g : P \rightarrow Q$}
		\BIC{$e(g, p) : Q$}
	\end{prooftree}
	Let $f : (P \rightarrow Q) \rightarrow P$. Then an instance of the modus ponens rules could be
	\begin{prooftree}
		\AXC{$g : P \rightarrow Q$}
		\AXC{$f : (P \rightarrow Q) \rightarrow P$}
		\BIC{$e(f, g) : P$}
	\end{prooftree}
	$e(f, g) \in Dom(g)$: A vicious circle!
\end{frame}


\begin{frame}{\subIdea}

	\begin{block}{Question}
		How to exclude the proofs of $P \rightarrow Q$ from possible proofs of $P$?

		\quad Let in only the proofs of $P$ that are constructed \emph{after} the proof of $P \rightarrow Q$.
	\end{block}

	So we need some notion of \emph{time} in our interpretation.

	\begin{itemize}
		\large
		\item A proof of $P \rightarrow Q$ \emph{at the time $t$}, is a function $f_t : p_s(P) \rightarrow p_s(Q)$, where $p_s(X)$ is the set of all proofs of the proposition $X$ \emph{at time $s$ after $t$}.
	\end{itemize}
\end{frame}

\begin{frame}{\subIdea}
	To implement this idea in a logical system, we can add a unary operator $\nabla$ to the language and interpret $\nabla P$ as the statement ``the proposition $P$ has been prooved some time in the past''.

	Then the modus ponens rule would look like this:
	
	\begin{prooftree}
		\AXC{$P$}
		\AXC{$\nabla (P \to Q)$}
		\BIC{$Q$}
	\end{prooftree}

	An implication can be used to deduced $Q$ from $P$ at present time, only if it has been proved some time in the past. Thus, its own proof is not a member of its domain anymore.
\end{frame}


\begin{frame}{\subSTL}
	The sequecnt calculus $\mathbf{LJ}$: {\small ($\Gamma$ and $\Sigma$ are multisets and $\Delta$ is a sub-singleton.)}
	\footnotesize
	\begin{multicols}{3}
		\begin{prooftree}
			\AXC{}
			\RightLabel{$Id$}
			\UIC{$A \Rightarrow A$}
		\end{prooftree}
		\columnbreak
		\begin{prooftree}
			\AXC{}
			\RightLabel{$L \bot$}
			\UIC{$ \bot \Rightarrow $}		
		\end{prooftree}
		\columnbreak
		\begin{prooftree}
			\AXC{}
			\RightLabel{$R \top$}
			\UIC{$ \Rightarrow \top$}
		\end{prooftree}
	\end{multicols}
	
	\begin{multicols}{3}
		\begin{prooftree}
			\AXC{$ \Gamma \Rightarrow \Delta$}
			\RightLabel{$L w$}
			\UIC{$ \Gamma, {A} \Rightarrow \Delta$}
		\end{prooftree}
		\columnbreak
		\begin{prooftree}
			\AXC{$ \Gamma \Rightarrow$}
			\RightLabel{$R w$}
			 \UIC{$\Gamma \Rightarrow {A}$}		
		\end{prooftree}
		\columnbreak
		 \begin{prooftree}
			 \AXC{$ \Gamma, A, A \Rightarrow \Delta$}
			 \RightLabel{$Lc$}
			 \UIC{$\Gamma, {A} \Rightarrow \Delta$}		
		 \end{prooftree}
	 \end{multicols}
	 
	\begin{prooftree}
		\AXC{$\Gamma \Rightarrow A$}
		\AXC{$\Sigma, A \Rightarrow \Delta$}
		\RightLabel{$cut$}
		\BIC{$\Gamma, \Sigma \Rightarrow \Delta$}
	\end{prooftree}
	\vspace{3mm}
	\begin{multicols}{3}
		\begin{prooftree}
			\AXC{$ \Gamma, A \Rightarrow \Delta$}
			\RightLabel{$L \wedge_1$}
			\UIC{$ \Gamma, {A \wedge B} \Rightarrow \Delta$}		
		\end{prooftree}
		\columnbreak
		\begin{prooftree}
			\AXC{$ \Gamma, B \Rightarrow \Delta$}
			\RightLabel{$L \wedge_2$}
			\UIC{$\Gamma, {A \wedge B} \Rightarrow \Delta$}		
		\end{prooftree}
		\columnbreak
		\begin{prooftree}
			\AXC{$\Gamma \Rightarrow A$}
			\AXC{$\Gamma \Rightarrow B$}
			\RightLabel{$R \wedge$}
			\BIC{$ \Gamma \Rightarrow {A \wedge B}$}
		\end{prooftree}
	\end{multicols}


		\begin{prooftree}
			\AXC{$ \Gamma, A \Rightarrow \Delta$}
			\AXC{$\Gamma, B \Rightarrow \Delta$}
			\RightLabel{$L \vee$}
			\BIC{$ \Gamma, {A \vee B} \Rightarrow \Delta$}

			\AXC{$\Gamma \Rightarrow A$}
			\RightLabel{$R \vee_1$}
			\UIC{$\Gamma \Rightarrow {A \vee B}$}

			\AXC{$\Gamma \Rightarrow B$}
			\RightLabel{$R \vee_2$}
			\UIC{$\Gamma \Rightarrow {A \vee B}$}

			\noLine\TIC{}
		\end{prooftree}

\end{frame}

\begin{frame}{\subSTL}
	
	Consider the sequent calculus $\mathbf{LJ}$ on the language $\mathcal{L} = \langle \vee, \wedge, \to, \nabla, \bot, \top \rangle$ and add the following rule to the system:
	\begin{prooftree}
		\AXC{$\Gamma \Rightarrow \Delta$}
		\RightLabel{$N$}
		\UIC{$\nabla \Gamma \Rightarrow \nabla \Delta$}
	\end{prooftree}
	Also replace the rules for implication with the following rules:
	\begin{columns}
		\column{0.5\textwidth}
		\begin{prooftree}
			\AXC{$\Gamma \Rightarrow A$}
			\AXC{$\Gamma, B \Rightarrow \Delta$}
			\RightLabel{$L \to$}
			\BIC{$\Gamma, \nabla (A \to B) \Rightarrow \Delta$}
		\end{prooftree}
		\column{0.5\textwidth}
		\begin{prooftree}
			\AXC{$\nabla \Gamma, A \Rightarrow B$}
			\RightLabel{$R \to$}
			\UIC{$\Gamma \Rightarrow A \to B$}
		\end{prooftree}
	\end{columns}
	\vspace{3ex}
	Call the resulting system $\mathbf{LDL}$. {\footnotesize (a.k.a. $\mathbf{iSTL(N)}$ in \cite{amir})}
\end{frame}

\begin{frame}[t]{\subSTL}
	Some important theorems in $\mathbf{LDL}$ \quad  $(\Box X = \top \to X)$
	\begin{block}{}
		\begin{columns}
			\column{0.5\textwidth}
			\[ \vdash \nabla \bot \Leftrightarrow \bot \]
			\column{0.5\textwidth}
			\[ \vdash \top \Leftrightarrow \nabla \top \]
		\end{columns}
		\vspace{1ex}

		\begin{columns}
			\column{0.5\textwidth}
			\[ \vdash \nabla (A \vee B) \Leftrightarrow \nabla A \vee \nabla B \]
			\column{0.5\textwidth}
			\[ \vdash \nabla (A \wedge B) \Leftrightarrow \nabla A \wedge \nabla B \]
		\end{columns}
		\vspace{1ex}

		\begin{columns}
			\column{0.5\textwidth}
			\[ \vdash \nabla \Box A \Rightarrow A \]
			\column{0.5\textwidth}
			\[ \vdash A \Rightarrow \Box \nabla A \]
		\end{columns}
		\vspace{1ex}

		\begin{columns}
			\column{0.5\textwidth}
			\[ \vdash A, \nabla (A \to B) \Rightarrow B \]
			\column{0.5\textwidth}
			\[ A \Rightarrow B \vdash \Rightarrow A \to B \]
		\end{columns}
		\begin{columns}
			\column{0.5\textwidth}
			\[ \nabla A \Rightarrow B \dashv\vdash A \Rightarrow \Box B \]
			\column{0.5\textwidth}
			\[ A, \nabla C \Rightarrow B \dashv\vdash C \Rightarrow A \to B \]
		\end{columns}
	\end{block}

	(Notice the adjunction relations.) 
\end{frame}

\section{\secSemantics}

\begin{frame}{\subAlgSemantics}
	\begin{block}{Definition}
		\begin{itemize}
			\item A \emph{locale} is a complete lattice whose meet distributes over all (possibly infinite) joins, or equivalently, a complete Heyting algebra.
			\item A pair $(X, \nabla)$ is called a \emph{dynamic locale}, if $X$ is a locale, and $\nabla : X \rightarrow X$ is a monotone operator that preserves finite meets and arbitrary joins.\\ {\footnotesize(introduced as ``$\nabla$-algebra'' in \cite{amir})}
		\end{itemize}
	\end{block}

	All dynamic locales have a \emph{weak implication} defined as follows. \[a \to b = \bigvee \{ x \mid a \wedge \nabla x \leq b \} \]
\end{frame}

\begin{frame}{\subAlgSemantics}
	Just like usual intuitionistic implication, there is an adjunction relation:
	\[ a \wedge \nabla x \leq y \iff x \leq a \to y \]

	Note that for any complete Heyting algebra $\mathcal{H}$, the pair $(\mathcal{H}, id_\mathcal{H})$ is a dynamic locale.
	
	Also one can observe that if $\mathcal{O}(X)$ is the locale of open sets of some topological space $X$, and $f : X \to X$ is a continuous function, then $(\mathcal{O}(X), f^{-1})$ is a dynamic local.
\end{frame}


\begin{frame}{\subAlgSemantics}
	We can interpret the set of $\mathcal{L}$-formulas $\Phi$ in a dynamic locale.

	\begin{block}{Definition}
		For any dynamic locale $S$, a \emph{valuation in $S$} is any map $V_S : \Phi \to S$ such that for all $a, b \in S$
		\begin{itemize}
			\item $V(a \vee b) = V(a) \vee V(b)$
			\item $V(a \wedge b) = V(a) \wedge V(b)$
			\item $V(a \to b) = V(a) \to V(b)$
			\item $V(\nabla a) = \nabla V(a)$
			\item $V(\bot) = 0$
			\item $V(\top) = 1$
		\end{itemize}
	\end{block}
\end{frame}

\begin{frame}{\subAlgSemantics}
	$\mathbf{LDL}$ is \emph{the logic of dynamic locale}.

	\begin{block}{Theorem}
		For any set of formulas $\Gamma \cup \{ A \}$ we have the following.
		\[ \bigwedge_{B \in \Gamma} V_S(B) \leq V_S(A) \text{~\small for any $V$ and $S$} \iff \mathbf{LDL} \vdash \Gamma \Rightarrow A \]

		That is, $\mathbf{LDL}$ is sound and complete with respect to the class of all dynamic locales. \cite{amir}
		\vspace{1ex}
	\end{block}
\end{frame}

\begin{frame}[t]{\subExtensions}
	\vspace{-1ex}
	\center
	\begin{multicols}{2}
		Additional rules:
		\columnbreak

		Corresponding dynamic locale axiom:
	\end{multicols}


	\begin{multicols}{2}
		\begin{prooftree}
			\RightLabel{$Fa$}
			\AXC{$\Gamma , A \Rightarrow B$}
			\UIC{$\Gamma \Rightarrow \nabla(A \to B)$}
		\end{prooftree}
		\columnbreak
		$\nabla$ is a surjection
	\end{multicols}

	\begin{multicols}{2}
		\begin{prooftree}
			\RightLabel{$Fu$}
			\AXC{$\nabla \Gamma \Rightarrow \nabla A$}
			\UIC{$\Gamma \Rightarrow A$}
		\end{prooftree}
		\columnbreak
		$\nabla$ is an injection
	\end{multicols}

	\begin{multicols}{2}
		\begin{prooftree}
			\RightLabel{$L$}
			\AXC{$\Gamma, A \Rightarrow \Delta$}
			\UIC{$\Gamma, \nabla A \Rightarrow \Delta$}
		\end{prooftree}
		\columnbreak
		$\nabla a \leq a$, for all $a$
	\end{multicols}

	\begin{multicols}{2}
		\begin{prooftree}
			\RightLabel{$R$}
			\AXC{$\nabla \Gamma, \Sigma \Rightarrow \Delta$}
			\UIC{$\Gamma, \Sigma \Rightarrow \Delta$}
		\end{prooftree}
		\columnbreak
		$a \leq \nabla a$, for all $a$
	\end{multicols}
\end{frame}

\begin{frame}{\subKripSemantics}
	$\mathcal{K}=(W, \leq, R, V)$ where
	\begin{itemize}
		\item $(W, \leq)$ is a poset
		\item $R \subseteq W \times W$
		\item If $(u, v) \in R$ and $u' \leq u$ and $v \leq v'$ then $(u', v') \in R$
	\end{itemize}
	$V: At(\mathcal{L}) \to U((W, \leq))$ where $At(\mathcal{L})$ is the set of atomic formulas and $U((W, \leq))$ is the set of all upsets of $(W, \leq)$.

	Define the forcing relation as usual using the relation $R$ for $\to$.
	
	\begin{block}{}
		$u \Vdash \nabla A$ if there exists $v \in W$ such that $(v, u) \in R$ and $v \Vdash A$.		
	\end{block}
	
	
	A Kripke model is called normal if there exists an order preserving function $\pi : W \to W$ such that $(u, v) \in R$ iff $u \leq \pi(v)$. If this $\pi$ exists, it would be unique.
\end{frame}

\begin{frame}{\subKripSemantics}
	\begin{block}{}
		Finally, a sequent $\Gamma \Rightarrow \Delta$ is valid in a Kripke model if for all $w \in W$, $w \Vdash \bigwedge \Gamma$ implies $w \Vdash \bigvee \Delta$.
	\end{block}\quad
	\vspace*{5mm}
	\begin{exampleblock}{Theorem}
		The logic $\mathbf{LDL}$ is sound and complete with respect to all normal Kripke models.
	\end{exampleblock}
\end{frame}

\section{\secProposal}

\begin{frame}{\subAnalytic}
	To facilitate the study from the perspective of proof-theory, a logical system must be analytic.

	\begin{block}{Definition}
		a sequent-style system \emph{analytic} if in all of its rules, any formula in the assumptions above the inference line is a sub-formula of some formula below the inference line.
		\vspace{1ex}
	\end{block}

	$\mathbf{LDL}$ is not analytic; it has a $cut$ rule.
	\begin{block}{}
		\begin{prooftree}
			\AXC{$\Gamma \Rightarrow A$}
			\AXC{$\Sigma, A \Rightarrow \Delta$}
			\RightLabel{$cut$}
			\BIC{$\Gamma, \Sigma \Rightarrow \Delta$}
		\end{prooftree}
	\end{block}
\end{frame}

\begin{frame}{\subGSTL}
	Remove the rules $cut$ and $Lc$, and alter the following rules in $\mathbf{LDL}$ to get $\mathbf{LDL3}$:
	\begin{block}{}
		\begin{multicols}{2}
			\begin{prooftree}
				\AXC{}
				\RightLabel{$Ex$}
				\UIC{$\nabla^n \bot \Rightarrow$}
			\end{prooftree}
			\columnbreak
			\begin{prooftree}
				\AXC{$\Gamma \Rightarrow \Delta$}
				\RightLabel{$Lw$}
				\UIC{$\Sigma, \Gamma \Rightarrow \Delta$}
			\end{prooftree}
		\end{multicols}

		\begin{multicols}{2}
			\begin{prooftree}
				\AXC{$\Gamma, \nabla^n A \Rightarrow \Delta$}
				\RightLabel{$L \wedge_1$}
				\UIC{$\Gamma, \nabla^n (A \wedge B) \Rightarrow \Delta$}
			\end{prooftree}
			\columnbreak
			\begin{prooftree}
				\AXC{$ \Gamma, \nabla^n B \Rightarrow \Delta$}
				\RightLabel{$L \wedge_2$}
				\UIC{$\Gamma, \nabla^n (A \wedge B) \Rightarrow \Delta$}
			\end{prooftree}
		\end{multicols}

		\begin{prooftree}
			\AXC{$\Gamma, \nabla^{n+1} (A \to B) \Rightarrow \nabla^n A$}
			\AXC{$\Gamma, \nabla^{n+1} (A \to B), \nabla^n B \Rightarrow \Delta$}
			\RightLabel{$L \to$}
			\BIC{$\Gamma, \nabla^{n+1} (A \to B) \Rightarrow \Delta$}
		\end{prooftree}

		\begin{prooftree}
			\AXC{$\Gamma, \nabla^n A \Rightarrow \Delta$}
			\AXC{$\Gamma, \nabla^n B \Rightarrow \Delta$}
			\RightLabel{$L \vee$}
			\BIC{$\Gamma, \nabla^n (A \vee B) \Rightarrow \Delta$}
		\end{prooftree}
	\end{block}
	The rules $cut$ and $Lc$ are admissible.
\end{frame}

\begin{frame}{\subResults}
	The results we have obtained for $\mathbf{LDL3}$ so far:
	\begin{itemize}
		\item Subformula property modulo $\nabla$
		\item Weak Deduction Theorem
		\item Admissibility of the Visser's rules
		\item Disjunction property
		\item Craig's interpolation property (for an extension with Heyting implication)
	\end{itemize}
\end{frame}

\begin{frame}{\subResults}
	\begin{exampleblock}{Subformula Property modulo $\nabla$}
		Any formula in any of the premises are of the form $\nabla^n A$ for some $n \geq 0$, where $A$ is a subformula of the conclusion.
	\end{exampleblock}
	\vspace{15mm}
	Rcall that some rules in $\mathbf{LDL3}$ have arbitrary $\nabla$s in their premises.
	
	This result will help us with the decidability of the system.
\end{frame}

\begin{frame}{\subResults}
	\begin{block}{Variant}
		For some fomula $A$, a variant of $A$ is either $A$ itself or $\nabla B$ or $\Box B$ for some variant of $A$ like $B$.
	\end{block}
	\begin{exampleblock}{Weak Deduction Theorem}
		The following are equivalent:
		\begin{enumerate}
			\item $\Rightarrow A \vdash_\mathbf{LDL3} \Gamma \Rightarrow \Delta$.
			\item $\mathbf{LDL3} \vdash \Gamma, V_A \Rightarrow \Delta$, for a set $V_A$ of variants of $A$.
		\end{enumerate}
	\end{exampleblock}
\end{frame}

\begin{frame}{\subResults}
	\begin{exampleblock}{Admissibility of the Generalized Visser's Rule}
		If $\mathbf{LDL3} \vdash \bigwedge_{j \in J} \nabla^{n_j} (A_j \rightarrow B_j) \rightarrow C \vee D$ then we have one of the following:
    \begin{enumerate}
      \item $\mathbf{LDL3} \vdash \bigwedge_{j \in J} \nabla^{n_i} (A_j \rightarrow B_j) \rightarrow \nabla^{n_k - 1} A_k$, {\small for some $k \in J$, where $n_k \neq 0$},
      \item $\mathbf{LDL3} \vdash \bigwedge_{j \in J} \nabla^{n_j} (A_j \rightarrow B_j) \rightarrow C$, or
      \item $\mathbf{LDL3} \vdash \bigwedge_{j \in J} \nabla^{n_j} (A_j \rightarrow B_j) \rightarrow D$
    \end{enumerate}
	\end{exampleblock}
	\begin{exampleblock}{Disjunction property}
		For all formulas $A$ and $B$, if $\mathbf{LDL} \vdash A \vee B$ then either $\mathbf{LDL} \vdash A$ or $\mathbf{LDL} \vdash B$.
	\end{exampleblock}
\end{frame}

\begin{frame}{\subResults}
	By $\mathbf{LDL}_\mathcal{H}$ we denote the system $\mathbf{LDL}$ over the language extended with the Heyting implication and augumented with its corresponding rules. By $V(A)$ we denote the set of atoms of $A$.
	\begin{exampleblock}{Craig's Interpolation Theorem}
		If $\mathbf{LDL}_\mathcal{H} \vdash A , \Rightarrow B$, then there is a formula $C$ such that:
  \begin{enumerate}
    \item $\mathbf{LDL}_\mathcal{H} \vdash A \Rightarrow C$,
    \item $\mathbf{LDL}_\mathcal{H} \vdash C \Rightarrow B$, and
    \item $V(C) \subseteq V(A) \cap V(B)$.
  \end{enumerate}
	\end{exampleblock}
\end{frame}

\begin{frame}{\subFurther}
	These goals are being pursued in proof-theoretic study of the Logic of Dyanmic Locales:
	\begin{itemize}
		\item Answering the question about the decidability of $\mathbf{LDL3}$
		\item Generalizing a similar idea about the universal quantifier,
		\item Desiginig a natural deduction version and a type theoretical counterpart for $\mathbf{LDL3}$.
	\end{itemize}
\end{frame}


\nocite{akbar2024generalization}
\nocite{akbar2024geometric}
\bibliographystyle{apalike}
\bibliography{references}

\begin{frame}[noframenumbering]{\quad}
	\begin{center}
		\Huge Thank you!
	\end{center}
\end{frame}

\end{document}
